\documentclass[12pt]{article}

\usepackage{graphicx}
\usepackage{geometry}
\geometry{margin=1in}
\usepackage{enumitem}
\usepackage{times}
\usepackage{titlesec}
\usepackage{float}
\usepackage{subcaption}
\usepackage{tabularx}
\titleformat{\section}{\large\bfseries}{\thesection}{1em}{}
\titleformat{\subsection}{\normalsize\bfseries}{\thesubsection}{1em}{}

\usepackage[colorlinks=true,  
            linkcolor=blue,
            urlcolor=cyan,  
            citecolor=blue]{hyperref}

\usepackage{setspace}
\usepackage{parskip}
\setlength{\parskip}{1em}

\begin{document}

\begin{titlepage}
    \begin{center}
        \begin{figure}[h]
            \centering
            \includegraphics[width=1\linewidth]{Header.png}
        \end{figure}
        
        \vspace{1cm}
        
        \textbf{1st Year Second Cycle} \\[0.5cm]
        {\LARGE\textbf{Introduction to Cyber Security Project}} \\[0.5cm]
        Groupe 07 \\[1cm]
        {\Huge\textbf{Edge Computing – Project Report}} \\[2cm]
        
        \textbf{Group Members:} \\[0.5cm]
        DOKKAR Chaima \\
        FERRAT Ilham \\
        KELLAKH Ranyme \\
        AIT HOCINE Khadidja \\
        KHEDRI Manel  \\
        BOUTRIA Manal \\
        BOUAKKAZ Madjeda \\
        DJEBOURI Lynda \\
        FERKIOUI Akram \\
        HAMMOUTI Walid \\[3cm]


    \end{center}
\end{titlepage}

\newpage
\tableofcontents
\newpage

\section{Introduction:}
IoT devices, mobile apps, and other systems that need super fast responses are everywhere
now. That’s why edge computing has become such a big deal, a system that moves data
processing and storage closer to where the data is generated instead of sending everything to
the cloud. This cuts down delays, saves bandwidth, and lets systems make decisions in real
time in areas like self-driving cars, smart cities, industrial automation, and healthcare
monitoring… But this setup also brings new security challenges since having devices and
servers all over the place makes them more vulnerable to attacks, especially since they don’t
have the same protections as big cloud servers.
Edge computing usually works in layers : the devices themselves (Edge Device Layer, EDL), the
local servers that support them (Edge Server Layer, ESL), and then the big cloud servers (Cloud
Server Layer, CSL). Devices like sensors and controllers interact directly with the real world,
handling data locally. That’s great for speed, but it also opens up new attack points. Edge
servers take on heavier tasks from devices, manage multiple endpoints, and make sure
communication stays secure. The cloud handles the big-picture stuff : analytics, AI, and storing
data long-term. Every layer helps the system work, but each also has its own weak spots.
Security in edge computing is tricky. Attacks like DDoS, side-channel exploits, malware, and
stolen credentials are common. Since edge systems are spread out, resource-limited, and
varied, old-school cloud security doesn’t always cut it. Edge computing needs smarter, lighter,
and adaptable defenses , ones that protect the system without slowing it down.
In this research, we’re looking at the architecture of edge computing, the security risks, and
privacy challenges, with a close look at each layer (EDL, ESL, CSL). The goal is to figure out
why problems happen, see how they’re currently handled, and point out big challenges for the
future.

\newpage
\section{Architecture of Edge Computing:}
In this section, we present a general architecture of edge computing shown in Figure
1, which mainly consists of three layers: an edge device layer (EDL), an edge server
layer (ESL), and a cloud server layer (CSL).
\subsection{Edge Device Layer (EDL):}
Edge devices are those low-level electronic devices deployed at EDL which operate
in the physical world to complete tasks such as sensing, actuating and controlling.
Each edge device is logically controlled by one or more microcontrollers (MCUs), with
each being a small computer running on a single integrated circuit . The low-level
software interface programmed in the MCUs that provide controls to the device’s
hardware is known as firmware. All the functions including sensing, controlling, and
computing are coded in the firmware and therefore, handled by the MCUs. Edge
devices can be further categorized as IoT devices and mobile devices. IoT devices
are lightweight electronic devices that are interconnected or connected to the edge
servers in ESL through wireless protocols such as 4G/5G, WiFi, and Bluetooth. They
usually run on lightweight preemptive /cooperative real-time operating systems
(RTOS), e.g., FreeRTOS and RT Thread . Once after a RTOS is burned into the chip
of the IoT devices, it usually does not provide further programming interfaces. Some
examples of IoT devices include : smart home devices, health monitoring devices,
and smart warehouse carts in industrialized IoT (IIoT). Different from IoT devices,
mobile devices usually have more advanced and costly preemptive operating
systems, e.g., Android and iOS, providing programmable interfaces for developers to
code their own applications at the top of the OSes. Some examples of mobile devices
include : smartphones, tablets, and central controllers of smart vehicles.
\subsection{Edge Server Layer (ESL):}
ESL has a hierarchical structure with multiple sub-layers consisting of various edge
servers with increasing computational power from bottom to up as shown in Figure *.
The edge servers located at the lowest sub-layer include wireless base stations and
access points (APs), which are mainly deployed for communication purpose to
receive data from the edge devices and send control flows back to them through
different wireless interfaces. Upon receiving data from edge devices, base
stations/APs forward the data to the edge servers located at the upper sub-layer,
which are mainly in charge of handling computation tasks. Upon receiving data
passed from base stations/APs or edge servers at the lower sub-layers, the edge
servers conduct relevant computation and analysis tasks on their own. If the
complexity of a task exceeds the computation limits of the current edge server, it
would offload the task to the servers located at the higher sub-layers, which possess
more powerful computation capabilities. These servers then conclude with asequence of control flows and pass them back to the base stations/APs, which
forward them to the edge devices in the end. Edge servers handle most of the core
computing functions such as authentication, authorization, computation, data
analytics, task offloading, and data storage for edge computing.Regardless of the
server's layer, there are other important components that play a major role in
connecting and communicating the sub-layers within this layer : the Edge Gateway
collects data from nearby IoT devices, performs basic processing, and forwards only
relevant information to upper layers. The Switch connects local devices (like edge
servers and gateways) within the same network, ensuring fast data transfer inside the
local area. The Router directs data between different networks ( for example, from
the edge network to the cloud ) managing traffic and choosing the best path for
communication. Together, they enable efficient, secure, and low-latency data flow
between devices and the cloud.
\subsection{Cloud Server Layer (CSL):}
Cloud server layer (CSL)hosts center cloud servers and data centers, with the cloud
servers responsible for the highest level authentication and authorization,
computation, and integration of different tasks offloaded from edge servers, and the
data centers in charge of storing vast amount of data generated by the edge devices
and edge servers. The state-of-the-art cloud servers and cloud data centers consist
of clusters of powerful machines. Because the security of CSL has been extensively
studied , in this article, we mainly explore the security issues of EDL and ESL in edge
computing.
\begin{figure}
    \centering
    \includegraphics[width=0.8\linewidth]{images/Architecture1.jpg}
    \caption{Architecture of Edge Computing}
\end{figure}
\subsection{Example}
\href{https://youtu.be/Abiu2YaY6b0?si=livQRpg5GOX9C_F9}{Source} \\
To better understand how communication and processing occur within this
architecture, let’s take one of the most well-known examples : the self-driving car.

1. Edge Devices Layer
In this layer, all sensors inside the self-driving car ( such as cameras, radar, LiDAR,
and GPS ) continuously collect real-time data about the surroundings. The onboard
computer processes this data locally and makes instant decisions, like braking or
steering, without sending everything to the cloud. This allows the car to react
immediately and avoid accidents by minimizing delay.

2. Edge Server Layer :
Here, nearby edge servers or 5G base stations receive summarized information from
several cars in the same area. These servers analyze local traffic conditions, detect
incidents, and send quick alerts back to nearby vehicles ( for example, warning them
about congestion or a roadblock ahead). This layer helps coordinate vehicles within a
region efficiently and with low latency.

3. Cloud Server Layer:
In the cloud layer, large data centers collect non-urgent information from many
vehicles and edge servers. The cloud performs advanced analytics, trains AI models
to improve driving algorithms, and sends software updates back to the cars. This
global processing ensures continuous learning and long-term improvement of
autonomous driving systems.
\begin{figure}[H]
    \centering
    \includegraphics[width=0.8\linewidth]{images/Architecture2.jpg}
    \caption{self-driving car}
\end{figure}

\newpage
\section{Edge Computing Security and Privacy:}
\subsection{DDoS Attacks and Defense Mechanisms in Edge Computing:}
\subsubsection{Introduction:}
DDoS refers to a type of active cyber attack in which external attackers aim to disrupt the availability of normal services provided by one or more servers using distributed resources such as a cluster of compromised edge devices (a.k.a. botnet). It mainly targets the Network, Transport, and Application layers of the OSI model. It is a powerful interruption attack that prevents legitimate users from accessing a service, thereby threatening the availability aspect of information security.
A traditional DDoS attack occurs when an attacker persistently sends massive streams of packets to a victim from compromised distributed electronic devices; thus, the hardware resources of the victim are quickly exhausted for handling these malicious packets and can no longer process legitimate requests on time. In other cases, attackers send malformed or spoofed packets that confuse the victim’s protocol, causing it to falsely conclude that all channels are occupied.
Edge servers are particularly vulnerable because they have limited computational capacity, weak authentication, and often operate with heterogeneous and insecure firmware. Attackers commonly compromise numerous edge devices and coordinate them remotely to launch large-scale attacks against edge infrastructures.

\subsubsection{Attack Specifications:}
In a typical DDoS scenario, the attacker (external) compromises a cluster of edge devices, gains full control, and then commands each to send flooding traffic toward the target edge server, overwhelming it until services shut down. Such flooding-based DDoS attacks include UDP flooding, ICMP flooding, SYN flooding, and HTTP flooding, all relying on saturating the victim’s network and processing capacity. Other variants such as zero-day attacks exploit unknown software vulnerabilities, affecting both availability and integrity of systems.
Defensive measures include per-packet inspection, traffic filtering, and rate limiting, though detection remains challenging due to IP spoofing and the resemblance between malicious and legitimate traffic.
An attacker in a DDoS scenario compromises a cluster of edge devices, takes control, and instructs them to send excessive traffic to a target edge server to overwhelm it and stop its service. Flooding-based attacks rely on high volumes of malformed or malicious packets and include major types: UDP flooding, ICMP flooding, SYN flooding, HTTP flooding, Ping of Death, and Slowloris. Brief mechanisms:
UDP flooding: Sends huge volumes of UDP packets to prevent the server from processing legitimate UDP traffic.
ICMP flooding: Sends many ICMP Echo Request (ping) packets rapidly; the victim replies to each, consuming both inbound and outbound throughput and slowing the system.
SYN flooding: Abuses TCP’s three-way handshake by sending many SYNs with spoofed IPs and not completing the handshake, leaving half-open connections on the server.
HTTP flooding: Sends a large number of legitimate-looking HTTP requests (GET/POST/PUT, etc.), overwhelming the server’s processing capacity.
Zero-day attacks: Exploit previously unknown software or protocol vulnerabilities (e.g., memory corruption) to crash or corrupt systems; these are hard to predict or filter because they abuse logic/implementation flaws rather than just volume.
\subsubsection{Defense Strategies Against Flooding:}
D1. Per-packet inspection and filtering: Drop suspicious packets before they reach the server (e.g., packet filters tied to congestion control). Effective but can be evaded by IP/MAC spoofing, forged headers, or tools that alter identifiers.
2. Statistics-based detection: Analyze traffic aggregates for anomalies (entropy measures, statistical clustering). Requires large traffic samples and struggles with encrypted flows.
3. Machine/deep learning: Use models (decision trees, Naive Bayes, autoencoders, neural nets) to detect attacks, including some encrypted traffic. These require large, representative training datasets, risk overfitting, and typically become effective only after substantial attack traffic has been observed.
Limitations: All approaches face practical challenges — IP spoofing, similarity between legitimate and attack traffic, encrypted flows that hinder inspection, and the need for extensive, high-quality training data. Effective defense usually combines real-time packet filtering with aggregate analysis and ML-based detection.
\subsubsection{Defense Strategies Against Zero-Day Exploits:}
Zero-day defenses focus on mitigating code-level vulnerabilities through memory safety analysis and protective isolation mechanisms.
Memory analysis and detection: Techniques like pointer-tainting, ECC memory, and firmware analysis help identify or reduce memory corruption. Deep learning models (RNNs, GNNs, NLP-based analyzers) can detect vulnerable code even without access to the source. However, encrypted firmware and anti-debugging measures often restrict their effectiveness.
Active protection and isolation: Methods such as in-process memory isolation, SDN-based IoT firewalls, lightweight isolation at access routers, and application-side fuzzing help minimize the attack surface or detect crashes. Despite their benefits, these methods can impose heavy computational overheads that are unsuitable for resource-constrained IoT or edge devices and may still fail to prevent complex zero-day exploits.
\begin{figure}[H]
    \centering
    \includegraphics[width=0.8\linewidth]{images/DDoS.jpg}
    \caption{A Typycal Architecture of a DDoS Attack}
\end{figure}

\subsection{Side Channel Attacks and Defenses:}

\subsubsection{Introduction:}
What if I told you hackers could steal your data without ever touching your files, cracking your password, or breaking into your system? Sounds impossible, right?  
That is exactly what happens in a side-channel attack — a method where attackers exploit indirect physical or system signals, such as power usage, timing, or data from system files like \texttt{/proc}, to extract secrets without breaking encryption or accessing the main code.

\subsubsection{OSI Layer Concerned:}
Side-channel attacks primarily target the \textbf{Physical Layer}, since they rely on observable physical behaviors like power consumption and electromagnetic signals.  
They can also affect the \textbf{Application Layer}, where exposed data sources and sensors provide additional leakage paths.

\subsubsection{Attack Type and Location:}
These attacks are generally \textbf{passive}, meaning the attacker observes system behavior rather than altering it.  
They can be launched by:
\begin{itemize}
    \item \textbf{Internal attackers:} malicious applications reading accessible data or sensors.
    \item \textbf{External attackers:} capturing power traces, electromagnetic emissions, or timing information remotely.
\end{itemize}

\subsubsection{Threat and Damage:}
Side-channel attacks primarily compromise \textbf{confidentiality}, exposing cryptographic keys, user data, or private operations.  
The potential damage can range from partial data leakage to a complete compromise of secure systems, depending on the sensitivity of the extracted information.

\subsubsection{Detection and Prevention:}
Detection of side-channel attacks is extremely difficult because they leave no direct logs or traces.  
Therefore, prevention focuses on reducing the available leakage sources through:
\begin{itemize}
    \item \textbf{Data Perturbation:} adding noise to timing, power, or communication patterns to hide correlations.
    \item \textbf{Access Control:} restricting access to system interfaces and files like \texttt{/proc}.
    \item \textbf{Obfuscation:} randomizing computations and introducing timing variations.
    \item \textbf{Hardware Isolation:} separating sensitive operations using trusted hardware zones.
\end{itemize}

\subsubsection{Attack Basis and Vulnerabilities:}
Side-channel attacks exploit indirect signals that correlate with internal system operations, such as:
\begin{itemize}
    \item Power consumption and timing delays.
    \item Electromagnetic emissions.
    \item System and sensor data exposure.
\end{itemize}
Edge computing devices are especially vulnerable due to their limited resources, high sensor density, and distributed nature, which increase both physical and software side-channel opportunities.

\subsubsection{Existing Solutions and Limitations:}
\begin{itemize}
    \item \textbf{Differential Privacy:} masks patterns with noise but reduces accuracy and data utility.
    \item \textbf{Access Control:} limits software-based leaks but cannot protect against physical emissions.
    \item \textbf{Obfuscation:} increases latency and energy usage.
    \item \textbf{Hardware Isolation:} enhances protection but cannot completely prevent signal leakage.
\end{itemize}
\subsection{Malware Injection:}
\subsubsection{Introduction:}
A malware injection attack occurs when malicious code is inserted into a legitimate process, service, or device
so that it executes within the victim’s environment. These attacks aim to compromise confidentiality, integrity, or
availability. They are particularly dangerous in edge computing because devices often have limited resources and
inconsistent patching mechanisms.
\subsubsection{Server-Side Injection Attacks:}
\begin{enumerate}
    \item \textbf{SQL Injection (SQLi):} OSI Layer: Application Attack Type: Active Attacker Location: External Attack Threat: Data confidentiality
and integrity Damage Level: High Detection Chance: Moderate Possibility of Prevention: High Attacks Based
On: Manipulating input fields to alter database queries Vulnerability: Dynamic SQL queries without proper san-
itization Existing Solutions and Limitations: Parameterized queries and ORMs mitigate risk but do not prevent
logic-level flaws.
    \item \textbf{Cross-Site Scripting (XSS):} OSI Layers: Presentation and Application Attack Type: Active Attacker Location: External Attack Threat:
Session hijacking and data theft Damage Level: Moderate to High Detection Chance: Low Possibility of Pre-
vention: Medium Attacks Based On: Injecting malicious JavaScript into web pages Vulnerability: Unsanitized
output rendered in browsers Existing Solutions and Limitations: Content Security Policy and escaping user input,
but developers often misapply them.
    \item \textbf{CSRF / SSRF:} OSI Layers: Network and Application Attack Type: Active Attacker Location: External Attack Threat: Unau-
thorized requests or internal resource access Damage Level: Moderate Detection Chance: Low Possibility of
Prevention: High Attacks Based On: Trick authenticated users or services into sending crafted requests Vul-
nerability: Missing CSRF tokens or improper request validation Existing Solutions and Limitations: SameSite
cookies and allowlists reduce exposure but may break service communication.
    \item \textbf{XML Signature Wrapping:} OSI Layers: Presentation and Application Attack Type: Active Attacker Location: External or Man-in-the-
Middle Attack Threat: Message integrity and authenticity Damage Level: High Detection Chance: Low Pos-
sibility of Prevention: High Attacks Based On: Wrapping signed XML nodes with new malicious elements
Vulnerability: Insecure XML parsing and ID binding Existing Solutions and Limitations: Schema validation
and correct parser configuration; however, legacy systems often skip updates.
\end{enumerate}
\subsubsection{Device-Side Injection Attacks:}
\begin{enumerate}
    \item \textbf{Firmware Injection:} OSI Layers: Physical to Application Attack Type: Active Attacker Location: Internal or Supply-Chain Attack
Threat: System takeover and persistent malware Damage Level: Critical Detection Chance: Very Low Possi-
bility of Prevention: Medium Attacks Based On: Modifying firmware updates or flashing malicious images
Vulnerability: Unsigned or unverified firmware Existing Solutions and Limitations: Secure boot and digital
signatures protect new devices but may not apply to legacy hardware.
    \item \textbf{Remote Code Execution (RCE):} OSI Layers: Transport and Application Attack Type: Active Attacker Location: External Attack Threat: Com-
plete device control Damage Level: Critical Detection Chance: Moderate Possibility of Prevention: High At-
tacks Based On: Exploiting vulnerabilities in running services Vulnerability: Outdated software or insecure
service exposure Existing Solutions and Limitations: Patching and sandboxing help but rely on consistent main-
tenance.
    \item \textbf{Supply-Chain Injection:} OSI Layer: Application Attack Type: Passive to Active Attacker Location: Internal / Third-Party Attack
Threat: Malicious dependency integration Damage Level: High Detection Chance: Low Possibility of Pre-
vention: Medium Attacks Based On: Inserting malware into libraries or software updates Vulnerability: Unver-
ified dependencies and unsigned packages Existing Solutions and Limitations: Code-signing and supply-chain
monitoring, but trust in third parties remains a weak link.
    \item \textbf{WebView / Mobile Injection:} OSI Layers: Presentation and Application Attack Type: Active Attacker Location: External Attack Threat:
Session hijacking and data manipulation Damage Level: Moderate Detection Chance: Low Possibility of Pre-
vention: Medium Attacks Based On: Abusing WebView or native-JS bridges in mobile apps Vulnerability:
Over-permissive WebView settings Existing Solutions and Limitations: Restrict bridge interfaces and sandbox
web content, though many apps neglect these controls.
\end{enumerate}
\subsubsection{Conclusion:}
Each malware injection vector targets different OSI layers and system components. Understanding which layer is
affected helps prioritize security controls and apply layered defenses across both server and device sides.

\subsection{Authentication and Authorization Attacks and Defense Mechanisms:}
\subsubsection{Introduction:}
Authentification is the process of verifying the identity of users who request certain
services.  Authorization is the process that determines an entity’s access rights and privileges.In edge computing, authentication often takes place between edge devices and edge
servers, but it can also be decentralized among devices.
\subsubsection{The four main types of authentification and authorisation attacks are:}

    Brute-force attacks ( guessing passwords to gain access), Authentication protocol flaws ( exploiting weaknesses), Authorization protocol flaws (bypassing access controls to gain privileges) and  Over-privilege attack ( abusing excessive permissions to perform unauthorized actions ).

\subsubsection{OSI layer targeted by the attack:}
In edge computing, authentication and authorization attacks mainly target the
application layer (identity, access, and API management).
They also affect the session ,presentation  and network layers .
\subsubsection{Attack type:}
Authentication and authorization attacks are generally active attacks, because the
attacker must interact directly with the system. They are rarely passive.
\subsubsection{Attack location:}
In edge computing, about 70\% of authentication and authorization attacks come from
external sources (through public interfaces ), while around 30\% are internal.
\subsubsection{Attack threat:}
Authentification and authorization attacks threaten several security properties: authentification , confidentiality , integrity and availability .

\subsubsection{Demage level:}
Authentication and authorization attacks can have high to critical impact, allowing
unauthorized access, data modification, and disruption of system operations. They
also undermine trust and accountability, making detection and response difficult.
\subsubsection{Reasons for low to moderate detection rate :}

  Attackers use legitimate credentials or sessions, blending with normal activity,
   Conventional monitoring cannot detect actions within granted privileges  and also 
  Effective detection requires advanced techniques (UEBA, AI-driven anomaly
detection).

\subsubsection{Preventive measures :}
 \textbf{Robust authentication:} Hardware-bound MFA, no password-only schemes , 
    \textbf{Granular authorization:} Least-privilege via dynamic RBAC/ABAC ,
 \textbf{Real-time monitoring:} UEBA, AI-driven anomaly detection, distributed SIEM and 
     \textbf{Zero-trust posture:} Verify all requests using mutual TLS and  workload identity.

\subsubsection{The authentification and authorisation attacks based on:}
The attack consists of impersonating a legitimate identity to perform unauthorized
actions in real time directly on an edge node.
It allows the attacker to steal, modify, redirect, or block critical flows (data,
commands) before they reach the cloud.
\subsubsection{The exploited vulnerabilities:}
 Weak identity, Poorly validated permissions ,
  Implicit network trust and 
    Exposed APIs without protection
    

\subsubsection{Exesting solutions and their limitations:}
\textbf{Existing solutions:} MFA (confirmation of user identity), RBAC/ABAC, end-to-end
encryption (mTLS) end API security.\\
\textbf{Limitations:} Vulnerable if credentials or tokens are stolen, nodes are compromised,
or policies misconfigured; they can add complexity, overhead, and scalability
challenges.

\subsection{A Comparison Study on Security Issues in Edge Computing and Cloud Computing}

\subsubsection{Architectural Differences Between Cloud and Edge Computing}

\begin{center}
\begin{tabular}{|p{3cm}|p{6cm}|p{6cm}|}
\hline
\textbf{Aspect} & \textbf{Cloud Computing} & \textbf{Edge Computing} \\
\hline
Architecture & Centralized — uses remote servers and large data centers on the Internet. & Decentralized — hierarchical structure with distributed edge servers. \\
\hline
End Devices & Fully-fledged computers (PCs, servers). & IoT and mobile devices with limited resources. \\
\hline
Connectivity & Mainly wired Internet connections. & Primarily wireless connections (Wi-Fi, 4G/5G). \\
\hline
Resource Capability & High — powerful centralized infrastructure. & Limited — low-profile edge devices and servers. \\
\hline
Data Processing Location & Centralized data processing in cloud data centers. & Localized processing near data sources (edge nodes). \\
\hline
Service Characteristics & High latency, not location-aware, large-scale centralized control. & Low latency, location-aware, real-time, privacy-enhanced, and bandwidth-efficient. \\
\hline
Attack Surface & Concentrated — fewer but high-value centralized targets. & Distributed — many small edge nodes vulnerable to local and physical attacks. \\
\hline
Security Challenges & Cloud provider security breaches, data leaks, insider threats, DDoS on central servers. & Node tampering, device hijacking, insecure wireless communication, local denial-of-service attacks. \\
\hline
Defense Focus & Strong centralized authentication, encryption, and access control. & Lightweight security, trust management, secure device authentication, decentralized intrusion detection. \\
\hline
Privacy & Centralized data storage — higher privacy risk. & Local data processing — improved privacy control. \\
\hline
\end{tabular}
\end{center}
\newpage
\subsubsection{Comparison of Security Attacks in Cloud and Edge Computing}

\begin{center}
\begin{tabular}{|p{4cm}|p{6cm}|p{6cm}|}
\hline
\textbf{Attack Type} & \textbf{Cloud Computing} & \textbf{Edge Computing} \\
\hline
DDoS Attacks & Uses heavyweight protocols (HTTP/HTTPS, FTP, SMTP) → harder to attack. Strong firewalls reduce impact. Still vulnerable to zero-day DDoS due to software flaws. & Uses lightweight protocols (CoAP, MQTT, UDP) → easier to attack. Low-end devices can be easily compromised (e.g., Mirai botnet). Flooding attacks still practically effective. \\
\hline
Side Channel Attacks & Wired connections limit attack surface. Attacks mostly through packet eavesdropping or crafted queries. & Wireless connectivity increases attack surface. Attackers can exploit wireless signals, power consumption, or proximity-based leaks. \\
\hline
Malware Injection Attacks & Attackers usually compromise the cloud server first, then infect clients. Harder to compromise cloud devices. & Attackers compromise edge devices first, then target servers. Interconnected IoT devices make propagation easier. \\
\hline
Authentication \& Authorization Attacks & Vulnerable to dictionary attacks and SSL/TLS weaknesses. Centralized → simpler authorization scenarios. & More weak passwords on IoT/mobile devices. Vulnerable to WPA/WPA2, Bluetooth, 4G/5G weaknesses. Decentralized → complex authorization, more prone to overprivilege attacks. \\
\hline
\end{tabular}
\end{center}

\subsubsection{Conclusion}

Cloud computing and edge computing offer similar services and functionalities, but their architectural differences lead to significant variations in security risks. Cloud computing, with its centralized design, powerful servers, and wired connections, is generally more resilient to traditional attacks such as DDoS and side channel exploits, though it remains vulnerable to zero-day attacks at the software level.

Edge computing, by contrast, relies on decentralized servers and resource-constrained IoT/mobile devices connected via wireless networks, which exposes it to a wider range of attacks, including device-level DDoS, side channel attacks, malware injection, and authentication/authorization exploits.

Overall, while cloud systems benefit from stronger protections and centralized management, edge systems trade off some security for proximity, real-time performance, and low-cost services, making security a more critical and complex challenge in edge computing environments.

\section{Root Causes, Grand Challenges , and Furure Research:}
\subsection{Root Causes of Security Issues:}
Current edge computing infrastructures suffer from serious cyber-attacks, which may
result in huge financial loss. The leading root causes of these security threats lie in
the unavoidable flaws or vulnerabilities in protocol and code designs as well as their
implementations, the concealed correlations between the public and the protected
private/secure data, and the lack of fine-grained access control.
\subsubsection{Protocol-Level Design Flaws:}
Protocol is the fundamental support
for basic functions such as communications and data processing in edge
computing from the theoretical perspective.Unfortunately, many of the current
protocols adopted by edge computing systems have design flaws because
their designers mainly focus on utility and user experience, treating security
as something unimportant if not unnecessary. By exploiting these flaws, an
attacker can achieve attack goals ranging from simply shutting down a normal
service (e.g. DDoS) to fully controlling an edge device or server (e.g.,
malware injection attacks).
\subsubsection{Implementation-Level Flaws:}
Implementation is the process of
practically realizing the edge computing functionalities. Even if a protocol may
be proven secure mathematically, it does not necessarily mean that its
implementation is secure. Logic flaws in an implementation can neutralize the
security strength of a protocol which has been proven strictly secure. There
are two main reasons leading to these flaws:
\begin{itemize}
    \item Developers may misunderstand the foundations of the protocol .
    \item Migrating a protocol from other platforms to the edge computing
platform may cause adaptivity in consistencies.
\end{itemize}
By exploiting implementation flaws, an attacker can bypass the security
features provided by the protocol (e.g., authentication attacks).
\subsubsection{Code-Level Vulnerabilities:}
Code is the basic unit of a program,
which defines the exact execution flow a processor should follow. Many
attacks are, in fact, the results of the code-level vulnerabilities. Note that
implementation-level flaws and code-level vulnerabilities are two different
concepts, with the former mainly referring to the logic flaws in a practical
realization while the latter mainly referring to the system bugs that can cause
memory failures/corruptions. Such vulnerabilities include stack/heap overflow,
use-after-free dangling pointer, format string, and so on. By exploiting these
vulnerabilities, an attacker can achieve attack goals ranging from simply
shutting down a normal service to fully controlling an edge device or server
(e.g., malware injection attacks).
\subsubsection{Data Correlations:}
In an edge computing system, tons of data would be
generated from the edge infrastructures constantly, with some of them being
sensitive and placed under strict protections while others less sensitive and
exposed to the public without protection. However, there may exist hidden
correlations between the sensitive data and insensitive data, which may not
be straightforward. Nevertheless, by exploiting these correlations and
leveraging various attack models, an attacker can infer the sensitive data
(e.g., side channel attacks) or even tamper them (e.g., bad data injection
attacks) based on the insensitive data.
\subsubsection{Lacking Fine-Grained Access Controls:}
Access control is one of
the most important security measures, via which an entity is permitted to
access only the least resources it needs. Such a measure has a good
reputation in protecting traditional general-purpose computing systems for a
long time but it cannot be directly adapted to edge computing due to the more
complex and fine-grained permission scenarios. Many current edge
computing systems implement only coarse grained access control
mechanisms, or even do not employ any access control mechanism. Lacking
a fine-grained access control specifically designed for edge computing
applications may significantly lower the bar of launching an attack (e.g.,
authorization attacks and MITM attacks).
\subsection{Status Quo and Grand Challenges:}
\subsubsection{Introduction:}
Edge computing has emerged to overcome the limitations of traditional cloud computing by providing low latency and localized processing. However, its rapid growth has outpaced the development of robust security mechanisms. Current systems face challenges in four key areas: design philosophy, framework adaptability, access control, and defensive capabilities.

\subsubsection{Lack of Security-by-Design:}
Edge systems are primarily designed for performance and low latency, often neglecting security considerations in the early stages. This results in built-in vulnerabilities that make edge nodes susceptible to attacks such as malware injection, DDoS, and unauthorized access. The distributed and heterogeneous nature of edge environments further complicates the later integration of security measures, highlighting the need for a security-by-design approach.

\subsubsection{Non-Migratability of Traditional Security Frameworks:}
Security solutions created for centralized cloud environments are not directly applicable to edge systems due to differences in computational capabilities, architectures, and communication protocols. The diversity of edge devices and lack of standardization cause fragmented security management and hinder the development of universal frameworks.

\subsubsection{Fragmented and Coarse-Grained Access Control:}
Access control in edge environments is inconsistent and lacks granularity. Different platforms use incompatible models, and permissions are often too broad to handle dynamic interactions between devices. Developing fine-grained, context-aware access control policies remains a significant research challenge.

\subsubsection{Isolated and Passive Defense Mechanisms:}
Current defense strategies are largely reactive and siloed, offering limited protection against advanced or coordinated attacks. Most rely on detection and patching rather than proactive threat prevention. A key challenge is to design adaptive and collaborative defense systems capable of anticipating and responding dynamically to evolving threats.
\subsection{Future Research Directions:}
\subsubsection{Introduction:}
Recent analyses of Distributed Denial-of-Service (DDoS) attacks, malware injection, and
authentication breaches reveal major limitations in current edge computing security. These
threats exploit edge-specific characteristics such as limited resources, protocol diversity, and
physical exposure. To strengthen resilience, future research must explore advanced,
adaptive defense mechanisms that evolve with emerging attack techniques.
\subsubsection{Multi-Layer Integrated Security Frameworks:}
Future studies should focus on designing comprehensive multi-layer architectures
addressing vulnerabilities across the edge stack. A promising direction is a three-layer
framework combining hardware-rooted trust, intelligent orchestration, and decentralized
management.\\
The outer layer enforces fine-grained, context-aware access control and dynamic
authorization, evaluating authentication based on user identity, location, and operational
context.\\
The middleware layer integrates software-defined networking (SDN) and network
function virtualization (NFV) enhanced with autonomous threat response. Using
federated learning, edge nodes collaboratively detect DDoS or anomaly patterns without
centralizing data, preserving bandwidth and privacy. Lightweight, unsupervised anomaly
detection models must be optimized for constrained edge resources while ensuring accuracy
against evolving malware.\\
The inner layer establishes hardware-based trust through Trusted Execution
Environments (TEEs), Physically Unclonable Functions (PUFs), and Memory
Protection Units (MPUs). These mechanisms safeguard sensitive operations in secure
enclaves and create isolation boundaries that resist firmware tampering and zero-day
attacks.
\subsubsection{Cross-Cutting Security Imperatives:}
Privacy and cryptographic resilience are key challenges. Homomorphic encryption and
differential privacy must be adapted for edge analytics to enable secure computation on
encrypted data without revealing sensitive information. With quantum computing on the
horizon, research should advance post-quantum cryptography and crypto-agility
frameworks, ensuring smooth algorithm migration across heterogeneous devices.Another priority is cross-layer vulnerability management—using formal verification to
assess risks from task offloading and maintaining consistent protection through unified
security policies. Scalable, automated patch management must ensure timely updates
across diverse edge ecosystems to prevent widespread exploitation.
\subsubsection{Implementation Challenges and Research Trajectories:}
Achieving these paradigms requires solving real-world challenges. The
performance-security tradeoff must be quantified, ensuring minimal latency impact for
real-time systems such as autonomous vehicles and industrial IoT. Research should also
emphasize interoperability standards, enabling consistent security across different edge
platforms, and energy-efficient protection mechanisms to suit battery-constrained
devices.
\subsubsection{Conclusion: Toward Comprehensive Edge Security:}
The evolution of edge computing security demands a shift from isolated countermeasures to
integrated, intelligent, and adaptive defense frameworks. Advancing research in
hardware-rooted trust, autonomous orchestration, privacy-preserving computation, and
cross-layer resilience will enable trustworthy, large-scale edge ecosystems. These
innovations are essential for the secure adoption of edge computing in critical domains such
as autonomous systems, industrial IoT, and smart cities.

\end{document}
